\begin{frame}
	\frametitle{SeismoCloud: obiettivo}
	
	\begin{itemize}
		\item<1-> Realizzare una rete di sismometri a basso costo
		\item<2-> Utilizzare la rete per individuare i terremoti \textbf{in tempo reale}
		\item<3-> Inviare un Early Warning, ovvero un avviso alle zone vicine all'epicentro, prima che il terremoto si propaghi
		\item<4-> Fornire alle persone (e macchine) un preavviso variabile da 2 a 20 secondi
		\item<5-> \textbf{NON} facciamo previsioni, \textbf{NON} cerchiamo precursori e \textbf{NON} possiamo far nulla per l'epicentro
	\end{itemize}
	
	Sito del Centro Nazionale Terremoti: http://cnt.rm.ingv.it/
	
\end{frame}
\begin{frame}
	\frametitle{SeismoCloud: Sembrano pochi secondi?}
	
	2-20 secondi possono sembrare pochi, ma dobbiamo considerare che:
	\pause
	\begin{itemize}
		\item<2-> Le persone impiegano del tempo (secondi a volte) per capire che si sta verificando un terremoto
		\item<3-> La scossa di terremoto è graduale, anche se veloce (secondi o millesimi di secondo)
		\item<4-> Gli edifici non si danneggiano (profondamente) subito (ma secondi dopo)
		\item<5-> L'obiettivo non è evacuare, ma portarsi in condizioni di sicurezza (eg. al riparo)
	\end{itemize}
	
\end{frame}
\begin{frame}
	\frametitle{SeismoCloud: mettersi al riparo}
	
	In 2-20 secondi possiamo:
	\pause
	\begin{itemize}
		\item<2-> Scendere da ponteggi o liberarsi dalle scale/ascensori
		\item<3-> Alzarci dal letto/divano (/svegliarsi) e guardarsi intorno per individuare un punto sicuro
		\item<4-> Iniziare a raggiungere un luogo dove ripararsi
		\item<5-> Se siamo al piano terra, fuggire all'esterno
		\item<6-> ...
	\end{itemize}
	
\end{frame}
\begin{frame}
	\frametitle{SeismoCloud: mettersi al riparo (2)}
	
	In 2-20 secondi un sistema elettronico/elettromeccanico può:
	\pause
	\begin{itemize}
		\item<2-> Mettere in sicurezza la distribuzione energetica/gas
		\item<3-> Bloccare l'ascensore al primo piano utile
		\item<4-> Mettere in sicurezza macchinari (in azienda) sensibili alle vibrazioni
		\item<5-> ...
	\end{itemize}
	
\end{frame}